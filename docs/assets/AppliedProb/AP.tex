% The current problem is sudo apt-get install texlive-fonts-extra

\documentclass[11pt]{scrartcl}
\usepackage[sexy]{evan}
% \usepackage{multirow}
\usepackage{array}
% \usepackage{program}
\usepackage{algorithm}
\usepackage{algpseudocode}
\usepackage{bbm}
\begin{document}

\title{Applied Probability}
\author{Arthur Conmy\footnote{Please send any corrections and/or feedback to \url{asc70@cam.ac.uk}}}
\date{Part II, Lent Term 2022}

\maketitle
\begin{abstract}
These notes are my best attempt at making a course with `applied' in the title i) exciting and ii) intuitive.
Credit  due to Evan Chen for the style file for these notes\footnote{Available here: \url{https://github.com/vEnhance/dotfiles/blob/master/texmf/tex/latex/evan/evan.sty}.}.
\end{abstract}

\begin{theorem}
\label{CTMC TFAE}
The following are equivalent:

\begin{itemize}
\item $(X_t)$ is a CTMC with generator $Q$.
\item $(X_t)$ satisfies the limiting transition property that 
\begin{itemize}
    \item When $y \neq x$, $\mathbb{ P}(X_{t+h} = y | X_t = x) = h q_{xy} + O(h^2)$.
    \item $\mathbb{ P}( X_{t+h} = x | X_t = x ) = 1 - h\sum_{y \neq x} q_{xy} + O(h^2)$.
\end{itemize}
\end{itemize}

\end{theorem}

\section{Reversibility}

What happens when we run CTMCs in reverse? We will restrict to the case $(Q, \pi)$ where we have an invariant distribution, much like we did in IB Markov Chains.

\begin{theorem}[Reversibility]
Let the irreducible, non-explosive CTMC $(X_t)$ with generator $Q$ have invariant distribution $\pi $. Then fixing some constant end time $T>0$, the random process $(\hat{X}_t) = (X_{T-t})$ for $0 \le t \le T$ is a CTMC, with invariant distribution 

\begin{equation}
    \hat{q}_{xy} = \frac{\pi_y}{\pi_x} q_{yx}.
\end{equation}
\end{theorem}

\begin{remark}
This is intuitively the case, since the distribution of $X_T$ will also follow the invariant distribution, and $\pi_x \hat{q}_{xy} = \pi_y q_{yx}$ encodes the fact that the reversed process will have transitions from state $x$ to state $y$ identical to the transitions from state $y$ to state $x$ in the original process. 
\end{remark}

We can bash this claim out with a lot of algebra, although building from the remark, we can see the result as a case of Bayes' theorem.

\begin{proof}
Since $\pi$ is invariant, $\forall t \in [0, T]$, $\hat{X}_t$ has distribution $\pi$. Now fix such a $t$ - we go for the second characterisation of (\ref{CTMC TFAE}). 

For $h$ small, and $y \neq x$, by Bayes' theorem,

\begin{equation}
    % h \hat{q}_{xy} :=
    \mathbb{ P}( \hat{X}_{t+h} = y | \hat{X}_t = x) = \mathbb{ P}( X_{T-t-h} = y | X_{T-t} = x ) = \frac{\mathbb{ P}(X_{T-t}=x | X_{T-t-h}=y) \mathbb{ P}(X_{T-t-h}=y)}{\mathbb{ P}(X_{T-t}=x)}
\end{equation}

which equals $\frac{h \pi_y q_{yx}}{\pi_x}$. Therefore indeed $\hat{X}$ is a CTMC.

\end{proof}


\end{document}